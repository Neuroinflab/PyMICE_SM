In this paper we have introduced PyMICE, a software library which allows to access
and analyze data from IntelliCage experiments. 
%
The library has been developed to 
facilitate automated, reproducible, and customizable analysis of large data 
generated by the IntelliCage system. Analyzer, the software bundled with the 
IntelliCage, does not meet these requirements, as it was designed with a 
different purpose in mind~\cite{intelliCagePlusManual2011}:
`The ``Analyzer'' is intended to give an overview of the results [...]
The function of ``Analyzer'' is to provide the user with data merging,
extraction, and filtering tools in order to generate data sets appropriate for
in-depth graphical and statistical analyses.'

One of the features of the IntelliCage system is that very different
experiments are possible, depending on the subject of the research.
Some protocols focus on assessment of subjects' ability of reward
location~\cite{Knapska:2013dj} and behavioral sequence~\cite{Endo:2011bs}
learning. Other protocols are dedicated to measure addiction-related
behavior like subject impulsiveness~\cite{Radwanska:2012fd,Mijakowska:2015io}.
Quite often a new experiment requires a
completely new approach to data analysis. Rather than trying to predict the
specific needs of the prospective users, we decided to provide simple,
intuitive and user-friendly interface for accessing the data. Such interface
allows a scientific programmer to tailor dedicated software focusing on the
essence of the analysis instead of the technical details. To our knowledge,
PyMICE is the only publicly available solution for analysis of IntelliCage 
data in a scripting language.

PyMICE is written in the Python programming language. Our choice of Python 
was directed by the same factors
which made it a popular choice for scientific computing in general. 
Python is free,
open-source, relatively easy to learn, and is supported by a number of
scientific tools and libraries, such as: NumPy and SciPy~\cite{Oliphant:2007ud},
IPython~\cite{Perez:2007wf}, matplotlib,
Pandas~\cite{mckinney-proc-scipy-2010}, etc.
We believe that PyMICE will be a useful addition to that collection.

The number of
IntelliCage-based publications is increasing in recent years~\cite{IntelliCageReferenceList},
but the system is still relatively little known. We believe that one of the factors 
handicapping the popularity of IntelliCage, or similar automated setups, is the lack of a proper 
software ecosystem. We hope that availability of PyMICE will have a stimulating
effect on the adoption of automated behavioral systems. 
While the current (at the time of the publication) version of the library  
only supports the IntelliCage, the library may be generalized to other behavioral systems. 
Data from any system capturing point events (such as
visits to specific locations -- as opposed to e.g. continuous trajectories of the animals) 
could be presented to the user in a similar way as the IntelliCage data. 
Specifically, representing each behavioral event as a Python object
with relevant attributes would allow for intuitive manipulation of data
and for easy extraction of the quantities which are analyzed. 
The PyMICE library is open source~\cite{gpl} and publicly available at
GitHub, the largest open source software platform~\cite{gousios2014}, 
therefore the extensions to other behavioral systems can be contributed by the community. 

A crucial feature of PyMICE is the possibility of creating automated
data analysis workflows. Such workflows are useful, for example, when the same 
protocol is applied to multiple groups of animals -- this is a very common case, 
as most \jknote{of?} experiments will have at least one experimental and one control group. 
A workflow defined in a Python script may be used to perform exactly the same
analysis on every available dataset, which both saves effort and greatly reduces
possibility of mistakes as compared to analyzing each dataset manually.

We also believe that popularization of such workflows would
lead to better research reproducibility. 
Current efforts for reproducible research are mostly focused on improving the
experimental procedures, statistical analysis, and the publishing
policy~\cite{begley2012,begley2013,halsey2015}. However, unclear or ambiguous 
description of data analysis is also given as a factor contributing to 
poor reproducibility of scientific research~\cite{ince2012}. A (non-interactive) 
computer program is a
precise, formal and unambiguous description of the analysis performed.
We hope that PyMICE could become a common platform for implementing
and sharing workflows for analysis of data from the IntelliCage (or similar) system, and
make data analysis using scripts more accesible and more popular.

The paper itself is a proof of the concept of the `really reproducible'
research~\cite{buckheit1995} -- writing it we followed the
literate programming paradigm~\cite{literateProgramming}. Every of the presented
results of analysis was generated by a Python + PyMICE workflow embedded in the
\LaTeX{}~\cite{latex} source code of the document (see the \emph{Statement of reproducibility} below for
details).

