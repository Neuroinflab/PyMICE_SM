%%%%%%%%%%%%%%%%%%%%%%% file template.tex %%%%%%%%%%%%%%%%%%%%%%%%%
%
% This is a general template file for the LaTeX package SVJour3
% for Springer journals.          Springer Heidelberg 2010/09/16
%
% Copy it to a new file with a new name and use it as the basis
% for your article. Delete % signs as needed.
%
% This template includes a few options for different layouts and
% content for various journals. Please consult a previous issue of
% your journal as needed.
%
%%%%%%%%%%%%%%%%%%%%%%%%%%%%%%%%%%%%%%%%%%%%%%%%%%%%%%%%%%%%%%%%%%%
%
% First comes an example EPS file -- just ignore it and
% proceed on the \documentclass line
% your LaTeX will extract the file if required
\begin{filecontents*}{example.eps}
%!PS-Adobe-3.0 EPSF-3.0
%%BoundingBox: 19 19 221 221
%%CreationDate: Mon Sep 29 1997
%%Creator: programmed by hand (JK)
%%EndComments
gsave
newpath
  20 20 moveto
  20 220 lineto
  220 220 lineto
  220 20 lineto
closepath
2 setlinewidth
gsave
  .4 setgray fill
grestore
stroke
grestore
\end{filecontents*}
%
\RequirePackage{fix-cm}
%
%\documentclass{svjour3}                     % onecolumn (standard format)
%\documentclass[smallcondensed]{svjour3}     % onecolumn (ditto)
\documentclass[smallextended]{svjour3}       % onecolumn (second format)
%\documentclass[twocolumn]{svjour3}          % twocolumn
%
\smartqed  % flush right qed marks, e.g. at end of proof
%
\usepackage{graphicx}
%
% \usepackage{mathptmx}      % use Times fonts if available on your TeX system
%
% insert here the call for the packages your document requires
%\usepackage{latexsym}
% etc.
\usepackage[T1]{fontenc}
\usepackage[utf8]{inputenc}
\usepackage{upquote,textcomp}
\usepackage{hyperref}
\newcommand{\code}[1]{\texttt{#1}}

% Insert the name of "your journal" with
% \journalname{myjournal}
%
\journalname{Behavior Research Methods}

\begin{document}

\title{PyMICE -- a Python library for analysis of IntelliCage data
%PyMICE -- facilitating reproducible research with IntelliCage system
%\thanks{Grants or other notes
%about the article that should go on the front page should be
%placed here. General acknowledgments should be placed at the end of the article.}
}
\subtitle{Supplementary materials}

\titlerunning{Supplementary materials for PyMICE -- a Python library...}        % if too long for running head

\author{Jakub M. Dzik \and
        Alicja Puścian \and
        Zofia Mijakowska \and
        Kasia Radwanska \and
        Szymon Łęski
}

%\authorrunning{Short form of author list} % if too long for running head

\institute{J. M. Dzik \and A. Puścian \and S. Łęski \at
              Department of Neurophysiology, Nencki Institute of Experimental Biology of Polish Academy of Sciences, Warsaw, Poland \\
%              Tel.: +123-45-678910\\
%              Fax: +123-45-678910\\
%              \email{fauthor@example.com}           %  \\
%             \emph{Present address:} of F. Author  %  if needed
           \and
           Z. Mijakowska \and K. Radwanska \at
              Department of Molecular and Cellular Neurobiology, Nencki Institute of Experimental Biology of Polish Academy of Sciences, Warsaw, Poland
           \and
           S. Łęski \at \email{s.leski@nencki.gov.pl}
}

\date{Received: date / Accepted: date}
% The correct dates will be entered by the editor


\maketitle

\begin{abstract}
We provide our readers with resources and guidance necessary to reproduce
the presented results.

%Insert your abstract here. Include keywords, PACS and mathematical
%subject classification numbers as needed.
\keywords{Reproducible research \and Literate programming \and Python scripts \and raw data}
% \PACS{PACS code1 \and PACS code2 \and more}
% \subclass{MSC code1 \and MSC code2 \and more}
\end{abstract}

%Your text comes here. Separate text sections with
\tableofcontents
\newpage

\section{Introduction}
In order to meet the standarts of reproducibility and enable a \emph{reproducibility
review}~\cite{peng2011} we decided to publish the data, scripts, and complete
source of the article~\cite{buckheit1995,literateProgramming}.  


\section{Requirements}
\input{supplementary_reproducibility}


\section{Reproduction of the examples}
\label{examples}

\subsection{Obtaining the example data}
\label{examples:data}
In order to save the example data in the working directory, run the
\emph{get\_example\_data.py} script:
\begin{verbatim}
$ python get_example_data.py
\end{verbatim}

However, the files are also available as supplementary materials (see
\ref{data:examples} for details).

\subsection{Example 1}
\label{examples:1}
In order to reproduce the 6-line example of data analysis, provide the
\emph{demo.zip} file (\ref{data:examples:demo}) in the working directory and
run the \emph{example1.py} script:
\begin{verbatim}
$ python example1.py
\end{verbatim}


\subsection{Example 2}
\label{examples:2}
To reproduce plot from \emph{Example 2},
provide the \emph{FVB} dataset (\ref{data:examples:fvb}) in the working
directory and run the \emph{example2.py} script:
\begin{verbatim}
$ python example2.py
\end{verbatim}


\subsection{Example 3}
\label{examples:3}
To reproduce plot from \emph{Example 3} (also figure 3A in~\cite{Puscian:2014cu}),
provide the \emph{C57\_AB} dataset (\ref{data:examples:c57}) in the working
directory and run the \emph{example3.py} script:
\begin{verbatim}
$ python example3.py
\end{verbatim}


\subsection{Example 4}
\label{examples:4}
To reproduce plot from \emph{Example 4} provide the \emph{C57\_AB}
dataset (\ref{data:examples:c57}) in the working directory and run the
\emph{example4.py} script:
\begin{verbatim}
$ python example4.py
\end{verbatim}


\section{Reproduction of the article}
\label{reproduction}
To automatically generate a custom version of the article based on reproduced
results rather than on the results claimed by the authors, perform the following
procedure: extract the contents of the
\emph{article\_source.zip} archive to the working directory.

In case the \emph{GNU Make} tool (or similar) is installed in the system, to
reproduce the article run:
\begin{verbatim}
$ make article.pdf
\end{verbatim}

If \emph{GNU Make} tool is not installed, you need to weave~\cite{literateProgramming}
the \emph{texw} files.
It is important to weave \emph{get\_example\_data.texw} as the first file,
since example data (\ref{data:examples}) are saved in the working directory
during its weaving.

\begin{verbatim}
$ Pweave -f tex get_example_data.texw
$ Pweave -f tex pymice_overview.texw
$ Pweave -f tex pymice_reproducibility.texw
$ Pweave -f tex example1.texw
$ Pweave -f tex -F figures example2.texw
$ Pweave -f tex -F figures example3.texw
$ Pweave -f tex -F figures example4.texw
\end{verbatim}

After all files have been weaved, compile the \LaTeX{} source.

\begin{verbatim}
$ pdflatex article.tex
$ bibtex article
$ pdflatex article.tex
$ pdflatex article.tex
\end{verbatim}

The \emph{article.pdf} file contains the reproduced article.


\section{Description of datasets}
\subsection{Datasets used in the examples}
\label{data:examples}


\subsubsection{demo.zip}
\label{data:examples:demo}
\emph{demo.zip} is a dataset prepared for demonstration purposes.
It is required by \emph{example1.py} script (\ref{examples:1}).


\subsubsection{FVB}
\label{data:examples:fvb}
The dataset concerning reward discrimination learning of FVB mice was obtained
during an ongoing research project and thus has not yet been published. For
that reason data were preprocessed to both anonimise it and make it more
suitable for demonstrational purposes. However, it should be stated that
presented data had actually been acquired with the use of FVB strain.

The dataset consist of one data file
and experiment timeline:
\begin{itemize}
\item \emph{2016-07-20~10.11.11.zip}
\item \emph{timeline.ini}
\end{itemize}
and is required by \emph{example2.py} (\ref{examples:2}).


\subsubsection{C57\_AB}
\label{data:examples:c57}
The \emph{C57\_AB} dataset was obtained during an actual research
project~\cite{Puscian:2014cu} and processed to make it suitable for more advanced
demonstrational purposes. It consist of four data files and experiment
timeline:
\begin{itemize}
\item \emph{2012-08-28~13.44.51.zip}
\item \emph{2012-08-28~15.33.58.zip}
\item \emph{2012-08-31~11.46.31.zip}
\item \emph{2012-08-31~11.58.22.zip}
\item \emph{timeline.ini}
\end{itemize}
and is required by 
\emph{example3.py} (\ref{examples:3}) and
\emph{example4.py} (\ref{examples:4}).


\subsection{Dataset discussed in the \emph{PyMICE library} section}
\emph{2014-09-11~14.13.34.zip} is a raw dataset obtained during an actual
research project by Maria Nalberczak from laboratory of KR.

The discussed 6-hour period starts at 22:00 (\mbox{UTC-08:00})
\mbox{2014-09-11}.
For mouse `9' the Analyzer software (v.~2.11.0.0) reports 480 licks less
than the number counted with PyMICE. The dataset has been inspected
manually. It seems that if the recorded start time of a nosepoke precedes the
recorded start time of the related visit, it will be omited by the Analyzer.

\begin{acknowledgements}
JD, KR and SŁ supported by a Symfonia NCN grant UMO-2013/\allowbreak 08/\allowbreak W/\allowbreak NZ4/\allowbreak 00691.
AP supported by a grant from Switzerland
through the Swiss Contribution to the enlarged European Union (PSPB-210/2010
to Ewelina Knapska and Hans-Peter Lipp).
KR and ZM supported by an FNP grant POMOST/2011-4/7 to KR.

%If you'd like to thank anyone, place your comments here
%and remove the percent signs.
\end{acknowledgements}

% BibTeX users please use one of
%\bibliographystyle{spbasic}      % basic style, author-year citations
\bibliographystyle{spmpsci}      % mathematics and physical sciences
%\bibliographystyle{spphys}       % APS-like style for physics
\bibliography{pymice}            % name your BibTeX data base


\end{document}
% end of file template.tex

