In recent years, a number of automated environments for behavioral testing
have been developed, based on
RFID~\cite{Dellomo:1998ts,Galsworthy:2005br,Daan:2011wl,EcoHABfullyauto:2015tk}
or video tracking of
animals~\cite{deChaumont:2012du,Weissbrod:2013bc,Shemesh:2013be,PerezEscudero:2014ed}.
Such automated systems have many advantages compared to the traditional behavioral tests,
such as the
reduction of the animal stress caused by isolation and
handling~\cite{Heinrichs:2006dk,Sorge:2014fb},
the possibility of studying social interactions in group-housed
animals~\cite{Kiryk:2011tk},
the possibility of long-term studies,
and easier standardization of protocols in turn leading to better
reproducibility of results between
laboratories~\cite{Crabbe:1999vf,Chesler:2002wo,Krackow:2010ck,Codita:2012jv,Morrison:2014cn,Vannoni:2014jt}.

The system we are particularly interested in is the IntelliCage
system~\cite{intelliCagePlusManual2011,Kiryk:2011tk,Radwanska:2012fd,Puscian:2014cu,Mijakowska:2015io}
(see \fig{intellicageSystem}), which is increasingly popular in behavioral research on
rodents~\cite{IntelliCageReferenceList}.

The system outputs a large amount of data
describing the behavior of the mice in the conditioning corners of the cage.
A typical experiment yields
10,000--100,000 visits recorded during several weeks or months.
Such large data call for development of data analysis methods
and software. One way to address the need of data analysis is to develop a
dedicated application, preferably with a graphical user interface (GUI),
which would allow researchers to inspect the data and extract relevant quantities
interactively. In fact, such an application, called Analyzer, is
provided with the IntelliCage system. While an
interactive GUI-based program for data analysis may be
useful, it does have certain limitations. In the context of scientific research,
perhaps the most severe limitation is poor reproducibility of the analysis,
unless strict measures are taken to record
every single action of the user. Moreover, there is usually no automated
way to perform exactly the same analysis on a different dataset,
and repeating the analysis manually is very time-consuming
and highly error-prone. Another inconvenience of ready-made programs is
that they are typically limited to a predefined set of analysis methods,
and not easily extendable.

Custom data analysis programs (e.g., scripts) fall at the opposite end of the
spectrum than GUI programs. First, the most obvious advantage of such
programs is the essentially unlimited possibility of implementing specialized
analysis methods and applying them much better (in terms of calculation speed,
precision and robustness) than a human.

Second, a noninteractive program (written in any programming language)
running in batch mode is, by
definition~\cite{Hoare69anaxiomatic,Turing1936,Floyd1967Flowcharts,Mccarthy63abasis},
an exhaustive specification of the analysis. In contrast, a plain language
description usually included in a Methods section of a journal article
may be ambiguous or not up-to-date.
The voices calling for providing the data analysis workflows together with
journal publications date at least two decades back~\cite{buckheit1995}:
`an article about computational science in a scientific publication is not
the scholarship itself, it is merely advertising of the scholarship. The
actual scholarship is the complete software development environment and the
complete set of instructions which generated the figures'.

Finally, one of the advantages of using an automated behavioral setup
is the possibility of high-throughput screening
by running the same protocol for a number of mice
cohorts (for example treatment and control groups and/or different strains
of mice~\cite{Puscian:2014cu}). Manual processing of each dataset separately
is both tedious and prone to errors. Batch-processing using a data
analysis script is an obvious remedy, as it allows for repetition of exactly
the same steps of analysis.

The only drawback is that the entry threshold for data analysis using a
programming language is much higher, as significant effort is required to
learn the programming language and necessary technical details (e.g., the data
format).
Our goal here is to lower this threshold by providing an easy and intuitive
way to access the IntelliCage data in the Python programming language~\cite{rossum1995}.

This paper is organized as follows: first, we briefly describe the IntelliCage system.
Next, we introduce PyMICE through a series of examples and provide pointers
to further documentation. We conclude with a short section on technical details and
a discussion.
