In this section, we introduce PyMICE through a series of examples illustrating various aspects of the library.\\
In Example 1 we show how to find numbers of visits of a specific mouse in which the first nosepoke was performed to either
the left side or the right side of the corner.
This can be achieved in PyMICE in just six lines of code. \\
Example 2 is an extension of Example 1 to analysis
of actual experimental data, obtained with a protocol described in~\cite{Knapska:2013dj}.
In this example, we also present a convention for
defining the timeline of the experiment.
\\
In Example 3 we reproduce a plot from an earlier paper~\cite{Puscian:2014cu}.
The plot shows how two cohorts of mice learn the location
of the reward over time (place preference learning). This kind of analysis
can be performed using Analyzer, the GUI application provided
with IntelliCage; however, using PyMICE we can quickly repeat the analysis for
new cohorts with minimal effort. \\
Example 4 illustrates how the Python programming language can be used
to extend the repertoire of data analysis methods.
In this example we show how to extract the information about intervals between visits of different mice to the same corner.
This kind of information would be very hard (or even impossible) to obtain using Analyzer.

To improve readability of Examples 2--4, we have omitted some generic code
and focused on PyMICE-specific snippets. Full code of the examples
is provided as online supplementary material at \url{https://github.com/Neuroinflab/PyMICE_SM/tree/examples}.

\input{get_example_data}

In addition to the examples presented here, we have prepared several
tutorials available online at the PyMICE website~\cite{pymiceWebsite}.
The tutorials are in Jupyter Notebook~\cite{jupyterOrg} format and may be downloaded
for interactive use.
The examples and the tutorials are provided as a hands-on introduction to
PyMICE and serve as a starting point for further exploration.
Additionally, online documentation is provided~\cite{pymiceDoc}. PyMICE
objects and their methods are also documented with docstrings available with
Python built in \code{help()} function.

